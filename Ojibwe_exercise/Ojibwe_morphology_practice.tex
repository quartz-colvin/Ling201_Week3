\documentclass[a4paper,11pt]{article}
\usepackage[margin=1in]{geometry}
\usepackage{enumerate}
\usepackage{xcolor}
 \usepackage{qtree}
\usepackage{fge}
\usepackage{tgtermes}

\usepackage{tikz}
\usepackage{tikz-qtree}

\usepackage{amsmath,amsthm, amssymb, latexsym}
\usepackage{graphicx, fancyhdr}
 \usepackage{stmaryrd}
 \usepackage{mathrsfs}
\usepackage{soul}
 \usepackage[utf8]{inputenc}
\usepackage[T1]{fontenc}

\usepackage{ot-tableau}
\usepackage{multicol}
\usepackage{enumitem}

\usepackage{tipa}
\usepackage{tipx}

\usepackage[sc]{mathpazo}
\usepackage[scaled=0.95]{helvet}
\usepackage{courier}
\usepackage{tgpagella}
\usepackage[euler]{textgreek}
 
\usepackage{gb4e}
\pagestyle{fancy}
\fancyhf{}
\fancyhead[C]{Ling201 \hfill 2025}
\fancyfoot[C]{\thepage}

\renewcommand{\theenumi}{\Alph{enumi}}


\begin{document}

\begin{center}
\Large{\textbf{Morphological analysis}}
\end{center}


\section{Nouns - Ojibwe (Algonquian)}

Ojibwe is a severely endangered Algonquian language spoken in the Great Lakes region and Canada. It's also the 4th most widely spoken Indigenous language in the US and the 2nd most spoken Indigenous language in Canada. 

It's an agglutinative language, and most "sentences" are conveyed by one really long verb. But the language can also form very complex nouns, which this dataset explores. In fact, the language-internal name for the language \textit{Anishinaabemowin} shows how complex nouns can be formed!

\begin{exe}
\ex \gll anishinaabe- mo- win \\
	native.person- speak.language- nominalizer \\
\glt \textit{"speaking the native language"}
\end{exe}




\subsection{Nouns: singular and plural}


\begin{multicols}{4}
\begin{exe}
\ex name 
\ex ogima: 
\ex o:d{\textyogh}i: 
\ex emikwa:n
\ex wa:ka:{\textglotstop}igan 
\ex amik
\end{exe}
\columnbreak
\begin{itemize}
\item[] \textit{"sturgeon"}
\item[] \textit{"chief"}
\item[] \textit{"fly"}
\item[] \textit{"spoon"}
\item[] \textit{"house"}
\item[] \textit{"beaver"}
\end{itemize}
\columnbreak
\begin{exe}
\ex {\textyogh}i:\textipa{S}i:b
\ex e:siban 
\ex namewag
\ex ogima:g 
\ex o:d{\textyogh}i:g 
\ex manido:g 
\end{exe}
\columnbreak
\begin{itemize}
\item[] \textit{"duck"}
\item[] \textit{"raccoon"}
\item[] \textit{"sturgeons"}
\item[] \textit{"chiefs"}
\item[] \textit{"flies"}
\item[] \textit{"spirits"}
\end{itemize}
\end{multicols}

\vspace{1cm}

\begin{itemize}
\item \underline{What is the Ojibwe plural morpheme?}
\item[] 
\item \underline{List all the roots.}
\end{itemize}

\pagebreak



\subsection{Diminutive \& augmentative nouns}

\textbf{Diminutive} morphemes are added to words to add a meaning of smallness, youngness, affection, or endearment to the original meaning of the word. \textbf{Augmentative} morphemes do the opposite. They add meanings such as bigness or a more intense sense of the original word. 

\vspace{1cm}

\begin{multicols}{2}
\begin{exe}
\ex e:mikwa:ne:ns 
\ex wi:giwa:me:ns 
\ex ase:ma:ns 
\ex manido:ns 
\ex i\textipa{S}kode:ns
\ex o:d{\textyogh}i:ns 
\ex {\textyogh}i:\textipa{S}i:be:ns 
\end{exe}
\columnbreak
\begin{itemize}
\item[] \textit{"little spoon"}
\item[] \textit{"little house"}
\item[] \textit{"cigarette"}
\item[] \textit{"bug/flying insect"}
\item[] \textit{"match"}
\item[] \textit{"little fly"}
\item[] \textit{"little duck"}
\end{itemize}
\end{multicols}

\begin{multicols}{2}
\begin{exe}
\ex git\textipa{S}ie:siban 
\ex git{\textyogh}iogima: 
\ex gi\textipa{S}imanido: 
\end{exe}
\columnbreak
\begin{itemize}
\item[] \textit{"big raccoon"}
\item[] \textit{"great leader"}
\item[] \textit{"great spirit"}
\end{itemize}
\end{multicols}

\vspace{1cm}

\begin{itemize}
\item \underline{What is the Ojibwe diminutive morpheme? List any allomorphs of the diminutive.}
\item[] 
\item \underline{What is the Ojibwe augmentative morpheme?}
\item[] 
\end{itemize}



\pagebreak
\subsection{Possession}



\begin{multicols}{2}
\begin{exe}
\ex nigo:ko:\textipa{S}im 
\ex niname:m 
\ex ni{\textyogh}i:\textipa{S}i:bim 
\ex no:s 
\ex no:komis 
\ex ni:ka:nis 
\ex nidakwe:m 
\ex nidogima:im 
\end{exe}
\columnbreak
\begin{itemize}
\item[] \textit{"my pig"}
\item[] \textit{"my sturgeon"}
\item[] \textit{"my duck"}
\item[] \textit{"my father"}
\item[] \textit{"my grandmother"}
\item[] \textit{"my dear friend"}
\item[] \textit{"my wife"}
\item[] \textit{"my chief/leader"}
\end{itemize}
\end{multicols}


\begin{multicols}{2}
\begin{exe}
\ex go:komis 
\ex go:ka:nis
\ex go:ko:\textipa{S} 
\ex gidakwe:m 
\ex giwa:ka:{\textglotstop}igan 
\end{exe}
\columnbreak
\begin{itemize}
\item[] \textit{"your grandmother"}
\item[] \textit{"your friend"}
\item[] \textit{"your pig"}
\item[] \textit{"your wife"}
\item[] \textit{"your house"}
\end{itemize}
\end{multicols}

\vspace{1cm}
\begin{itemize}
\item \underline{What is the Ojibwe morpheme for \textit{"my"?}}
\item[] 
\item \underline{Is there more than one way to mark possession?}
\item[] 
\item \underline{What is the Ojibwe morpheme for \textit{"your"}?}
\item[] 
\end{itemize}



\pagebreak
\subsection{Adjectives?}

Other morphemes can be added, or \textit{incorporated} into nouns in Ojibwe, too. Here are some additional data points with adjective-like meanings within the noun.

\vspace{1cm}
\begin{multicols}{2}
\begin{exe}
\ex git\textipa{S}imad{\textyogh}ie:siban 
\ex wa:bamik
\ex makadewamik 
\ex ga:skamik
\end{exe}
\columnbreak
\begin{itemize}
\item[] \textit{"big, wicked raccoon"}
\item[] \textit{"albino beaver"}
\item[] \textit{"black beaver"}
\item[] \textit{"smoked beaver"}
\end{itemize}
\end{multicols}
\vspace{1cm}

\begin{itemize}
\item \underline{What are the Ojibwe morphemes for \textit{"wicked", "albino", "black", and "smoked"}?}
\end{itemize}
\vspace{4cm}

\begin{thebibliography}{20}
\addtolength{\leftmargin}{0.2in} % sets up alignment with the following line.
\setlength{\itemindent}{-0.2in}

\bibitem{Newell} Newell, H. \& Piggott, G. 2014. Interactions at the syntax-phonology interface: Evidence from Ojibwe. \textit{Lingua} 150 332-362.

\bibitem{Dictionary} The Ojibwe People’s Dictionary. (n.d.). https://ojibwe.lib.umn.edu/search?


\end{thebibliography}

\end{document}