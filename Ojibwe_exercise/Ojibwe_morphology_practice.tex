\documentclass[a4paper,11pt]{article}
\usepackage[margin=1in]{geometry}
\usepackage{enumerate}
\usepackage{xcolor}
 \usepackage{qtree}
\usepackage{fge}
\usepackage{tgtermes}

\usepackage{tikz}
\usepackage{tikz-qtree}

\usepackage{amsmath,amsthm, amssymb, latexsym}
\usepackage{graphicx, fancyhdr}
 \usepackage{stmaryrd}
 \usepackage{mathrsfs}
\usepackage{soul}
 \usepackage[utf8]{inputenc}
\usepackage[T1]{fontenc}

\usepackage{ot-tableau}
\usepackage{multicol}
\usepackage{enumitem}

\usepackage{tipa}
\usepackage{tipx}

\usepackage[sc]{mathpazo}
\usepackage[scaled=0.95]{helvet}
\usepackage{courier}
\usepackage{tgpagella}
\usepackage[euler]{textgreek}
 
\usepackage{gb4e}
\pagestyle{fancy}
\fancyhf{}
\fancyhead[C]{Ling201:02 \hfill 2025}
\fancyfoot[C]{\thepage}

\renewcommand{\theenumi}{\Alph{enumi}}


\begin{document}

\begin{center}
\Large{\textbf{Practice datasets}: Morphological analysis}
\end{center}


\section{Nouns - Anishinaabemowin (AKA Ojibwe, Algonquian)}

\textcolor{red}{background info on Ojibwe???}




\subsection{Nouns: singular and plural}


\begin{multicols}{4}
\begin{exe}
\ex name 
\ex ogima: 
\ex o:d{\textyogh}i: 
\ex emikwa:n
\ex wa:ka:{\textglotstop}igan 
\ex {\textyogh}i:\textipa{S}i:b
\ex e:siban 
\end{exe}
\columnbreak
\begin{itemize}
\item[] \textit{"sturgeon"}
\item[] \textit{"chief"}
\item[] \textit{"fly"}
\item[] \textit{"spoon"}
\item[] \textit{"house"}
\item[] \textit{"duck"}
\item[] \textit{"raccoon"}
\end{itemize}
\columnbreak
\begin{exe}
\ex name:g
\ex ogima:g 
\ex ma\textipa{S}kode:n 
\ex o:d{\textyogh}i:g 
\ex manido:g 
\end{exe}
\columnbreak
\begin{itemize}
\item[] \textit{"sturgeons"}
\item[] \textit{"chiefs"}
\item[] \textit{"firlds"}
\item[] \textit{"flies"}
\item[] \textit{"spirits"}
\end{itemize}
\end{multicols}

\vspace{1cm}

\begin{itemize}
\item \underline{What is the Ojibwe plural morpheme?}
\item[] 
\item \underline{List all the roots.}
\end{itemize}

\pagebreak



\subsection{Diminutive \& augmentative nouns}

\textcolor{red}{add a blurb about what diminutive and augmentative mean???}

\vspace{1cm}

\begin{multicols}{2}
\begin{exe}
\ex e:mikwa:ne:ns 
\ex wi:giwa:me:ns 
\ex ase:ma:ns 
\ex manido:ns 
\ex i\textipa{S}kode:ns
\ex o:d{\textyogh}i:ns 
\ex aga:\textipa{S}io:d{\textyogh}i: 
\ex {\textyogh}i:\textipa{S}i:be:ns 
\end{exe}
\columnbreak
\begin{itemize}
\item[] \textit{"little spoon"}
\item[] \textit{"little house"}
\item[] \textit{"cigarette"}
\item[] \textit{"bug/flying insect"}
\item[] \textit{"match"}
\item[] \textit{"little fly"}
\item[] \textit{"small fly"}
\item[] \textit{"little duck"}
\end{itemize}
\end{multicols}

\begin{multicols}{2}
\begin{exe}
\ex git\textipa{S}ie:siban 
\ex git{\textyogh}iogima: 
\ex git{\textipa{S}}iemikwa:nens 
\ex gi\textipa{S}imanido: 
\end{exe}
\columnbreak
\begin{itemize}
\item[] \textit{"big raccoon"}
\item[] \textit{"great leader"}
\item[] \textit{"big spoon"}
\item[] \textit{"great spirit"}
\end{itemize}
\end{multicols}

\vspace{1cm}

\begin{itemize}
\item \underline{What is the Ojibwe diminutive morpheme?}
\item[] 
\item \underline{What is the Ojibwe augmentative morpheme?}
\item[] 
\end{itemize}



\pagebreak
\subsection{Possession}

\begin{multicols}{2}
\begin{exe}
\ex nigo:ko:\textipa{S}im 
\ex niname:m 
\ex ni{\textyogh}i:\textipa{S}i:bim 
\ex no:s 
\ex no:komis 
\ex ni:ka:nis 
\ex nidakwe:m 
\ex nidogima:im 
\end{exe}
\columnbreak
\begin{itemize}
\item[] \textit{"my pig"}
\item[] \textit{"my sturgeon"}
\item[] \textit{"my duck"}
\item[] \textit{"my father"}
\item[] \textit{"my grandmother"}
\item[] \textit{"my dear friend"}
\item[] \textit{"my wife"}
\item[] \textit{"my chief/leader"}
\end{itemize}
\end{multicols}


\begin{multicols}{2}
\begin{exe}
\ex go:komis 
\ex go:ka:nis
\ex go:ko:\textipa{S} 
\ex gidakwe:m 
\end{exe}
\columnbreak
\begin{itemize}
\item[] \textit{"your grandmother"}
\item[] \textit{"your friend"}
\item[] \textit{"your pig"}
\item[] \textit{"your wife"}
\end{itemize}
\end{multicols}

\vspace{1cm}
\begin{itemize}
\item \underline{What is the Ojibwe morpheme for \textit{"my"?}}
\item[] 
\item \underline{What is the Ojibwe morpheme for \textit{"your"}?}
\item[] 
\end{itemize}



\pagebreak
\subsection{Adjectives?}

\textcolor{red}{add info on adjective incorporation???}

\vspace{1cm}
\begin{exe}
\ex git\textipa{S}imad{\textyogh}ie:siban \hspace{2cm} \textit{"big, wicked raccoon"}
\end{exe}
\vspace{1cm}

\begin{itemize}
\item \underline{What is the Ojibwe morpheme for \textit{"wicked"?}}
\end{itemize}
\vspace{1cm}

\begin{thebibliography}{20}
\addtolength{\leftmargin}{0.2in} % sets up alignment with the following line.
\setlength{\itemindent}{-0.2in}

\bibitem{Newell} Newell, H. \& Piggott, G. 2014. Interactions at the syntax-phonology interface: Evidence from Ojibwe. \textit{Lingua} 150 332-362.

\bibitem{Dictionary} The Ojibwe People’s Dictionary. (n.d.). https://ojibwe.lib.umn.edu/search?


\end{thebibliography}

\end{document}